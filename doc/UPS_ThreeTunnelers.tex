\documentclass[12pt,a4paper]{article}
\usepackage[utf8]{inputenc}
\usepackage[czech]{babel}
\usepackage[T1]{fontenc}
\usepackage{amsmath}
\usepackage{amsfonts}
\usepackage{amssymb}
\usepackage{graphicx}
\usepackage{titlesec}
\usepackage[left=2cm,right=2cm,top=2cm,bottom=2cm]{geometry}
\usepackage{indentfirst}
\usepackage{listings}
\usepackage{color}
\usepackage{array}
\usepackage{csquotes}

%Pravidlo pro řádkování
\renewcommand{\baselinestretch}{1.5}

%Pravidlo pro začínání kapitol na novém řádku
\let\oldsection\section
\renewcommand\section{\clearpage\oldsection}

%Formáty písem pro nadpisy (-změněno na bezpatkové \sffamily z původního \normalfont
\titleformat{\section}
{\sffamily\Large\bfseries}{\thesection}{1em}{}
\titleformat{\subsection}
{\sffamily\large\bfseries}{\thesubsection}{1em}{}
\titleformat{\subsubsection}
{\sffamily\normalsize\bfseries}{\thesubsubsection}{1em}{}

%Nastavení zvýrazňování kódu v \lslisting
\definecolor{mygreen}{rgb}{0,0.6,0}
\definecolor{mygray}{rgb}{0.5,0.5,0.5}
\lstset{commentstyle=\color{mygreen},keywordstyle=\color{blue},numberstyle=\tiny\color{mygray}}

\author{Jan Šmejkal}

\begin{document}

%-------------Úvodni strana---------------
\begin{titlepage}

\includegraphics[width=50mm]{img/FAV.jpg}
\\[160 pt]
\centerline{ \Huge \sc KIV/UPS - Úvod do Počítačových sítí}
\centerline{ \LARGE \sc Semestrální práce }
\\[12 pt]
{\Huge \sc
\centerline{Three Tunnelers}
}


{
\vfill 
\parindent=0cm
\textbf{Jméno:} Štěpán Ševčík\\
\textbf{Osobní číslo:} A13B0443P\\
\textbf{E-mail:} kiwi@students.zcu.cz\\
\textbf{Datum:} {\large \today\par} %datum

}

\end{titlepage}

%------------------------------------------

%------------------Obsah-------------------
\newpage
\setcounter{page}{2}
\setcounter{tocdepth}{3}
\tableofcontents
%------------------------------------------

%--------------Text dokumentu--------------


\section{Zadání}
\noindent
Počítačová hra Three Tunnelers založená na původní DOSové hře Tunnelers sestávající z:
\begin{itemize}
\setlength\itemsep{-1em}
\item Herní klient implementovaný v jazyce JAVA
\item Herní server implementovaný v jazyce C
\end{itemize}
\subsection*{Zásady vypracování semestrální práce viz. CW}
\begin{itemize}
\setlength\itemsep{0em}
\item Úlohu naprogramujte v programovacím jazyku C/C++ anebo Java. Pokud se jedná o úlohu server/klient, pak klient bude v Javě a server v C/C++.

\item Komunikace bude realizována textovým nešifrovaným protokolem nad TCP protokolem.

\item Výstupy serveru budou v alfanumerické podobě, klient může komunikovat i v grafice (není podmínkou).

\item Server řešte pod operačním systémem Linux, klient může běžet pod OS Windows XP. Emulátory typu Cygwin nebudou podporovány.

\item Realizujte konkurentní (paralelní) servery. Server musí být schopen obsluhovat požadavky více klientů souběžně.

\item Součástí programu bude trasování komunikace, dovolující zachytit proces komunikace na úrovni aplikačního protokolu a zápis trasování do souboru.

\item Každý program bude doplněn o zpracování statistických údajů (přenesený počet bytů, přenesený počet zpráv, počet navázaných spojení, počet přenosů zrušených pro chybu, doba běhu apod.).

\item Zdrojové kódy organizujte tak, aby od sebe byly odděleny části volání komunikačních funkcí, které jste vytvořili na základě zadání, od částí určených k demonstraci funkčnosti vašeho řešení (grafické rozhraní).
\end{itemize}
\section{Programátorská dokumentace}
\subsection{Protokol}
Zprávy mezi klientem a serverem se přenáší pomocí protokolu realizovaného nad transportním protokolem TCP. Zprávy nemusí mít jednotnou délku ale vždy musí odpovídat následujícímu formátu:
\begin{center}
\texttt{<hlavička><tělo><LF>}
\end{center}
\begin{itemize}
\setlength\itemsep{0em}
\item \textbf{hlavička} zprávy se skládá ze 4 hexadecimálních znaků, které představují typ zprávy
\item \textbf{tělo} zprávy se odvíjí podle typu zprávy, čísla jsou přenášena v hexadecimální podobě a jsou vycpávána nulami na začátku do konstantní délky
\item \textbf{LF} představuje znak "Line Feed" a je vyžadován na konci každé zprávy
\end{itemize}
\subsubsection{Výčet typů zpráv a jejich významů}
\begin{table}[h]
\center
\begin{tabular}{|c|c|c|p{10cm}|}
\hline
Kód & Název & Zdroj & Tělo zprávy / poznámka\\ \hline
0 & Nedefinováno & - & Používá se pro rozpoznání neplatného přenosu \\ \hline
1 & Potvrzení & K/S & Generické schválení požadavku, nepoužívat \\ \hline
2 & Zamítnutí & K/S & Generické zamítnutí požadavku, nepoužívat \\ \hline
3 & Echo Request & K/S & Vyžádání aktivity protější strany \\ \hline
4 & Echo Reply & K/S & Odpověď generující aktivitu \\ \hline
5 & Špatný formát & K/S & Oznámení protistraně o nerozpoznaném příkazu \\ \hline
6 & Představení & K/S & Představení klienta \newline Tělo zprávy od klienta nenese žádný význam \newline Tělo zprávy od serveru obsahuje klientské id tohoto klienta a tajný kód pro opětovné připojení \\ \hline
7 & Znovu-představení & K & Pokus o znovu navázání spojení pod stejnou identitou \\ \hline

\end{tabular}
\end{table}
\subsection{Herní server}
\subsection{Herní klient}

\section{Uživatelská dokumentace}




%------------------------------------------

\end{document}
