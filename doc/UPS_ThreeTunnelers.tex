\documentclass[12pt,a4paper]{article}
\usepackage[utf8]{inputenc}
\usepackage[czech]{babel}
\usepackage[T1]{fontenc}
\usepackage{amsmath}
\usepackage{amsfonts}
\usepackage{amssymb}
\usepackage{graphicx}
\usepackage{titlesec}
\usepackage[left=2cm,right=2cm,top=2cm,bottom=2cm]{geometry}
\usepackage{indentfirst}
\usepackage{listings}
\usepackage{color}
\usepackage{array}
\usepackage{csquotes}

%Pravidlo pro řádkování
\renewcommand{\baselinestretch}{1.5}

%Pravidlo pro začínání kapitol na novém řádku
\let\oldsection\section
\renewcommand\section{\clearpage\oldsection}

%Formáty písem pro nadpisy (-změněno na bezpatkové \sffamily z původního \normalfont
\titleformat{\section}
{\sffamily\Large\bfseries}{\thesection}{1em}{}
\titleformat{\subsection}
{\sffamily\large\bfseries}{\thesubsection}{1em}{}
\titleformat{\subsubsection}
{\sffamily\normalsize\bfseries}{\thesubsubsection}{1em}{}

%Nastavení zvýrazňování kódu v \lslisting
\definecolor{mygreen}{rgb}{0,0.6,0}
\definecolor{mygray}{rgb}{0.5,0.5,0.5}
\lstset{commentstyle=\color{mygreen},keywordstyle=\color{blue},numberstyle=\tiny\color{mygray}}

\author{Jan Šmejkal}

\begin{document}

%-------------Úvodni strana---------------
\begin{titlepage}

\includegraphics[width=50mm]{img/FAV.jpg}
\\[160 pt]
\centerline{ \Huge \sc KIV/UPS - Úvod do Počítačových sítí}
\centerline{ \LARGE \sc Semestrální práce }
\\[12 pt]
{\Huge \sc
\centerline{Three Tunnelers}
}


{
\vfill 
\parindent=0cm
\textbf{Jméno:} Štěpán Ševčík\\
\textbf{Osobní číslo:} A13B0443P\\
\textbf{E-mail:} kiwi@students.zcu.cz\\
\textbf{Datum:} {\large \today\par} %datum

}

\end{titlepage}

%------------------------------------------

%------------------Obsah-------------------
\newpage
\setcounter{page}{2}
\setcounter{tocdepth}{3}
\tableofcontents
%------------------------------------------

%--------------Text dokumentu--------------


\section{Zadání}
\noindent
Počítačová hra Three Tunnelers založená na původní DOSové hře Tunnelers sestávající z:
\begin{itemize}
\setlength\itemsep{-1em}
\item Herní klient implementovaný v jazyce JAVA
\item Herní server implementovaný v jazyce C
\end{itemize}
\subsection*{Zásady vypracování semestrální práce viz. CW}
\begin{itemize}
\setlength\itemsep{0em}
\item Úlohu naprogramujte v programovacím jazyku C/C++ anebo Java. Pokud se jedná o úlohu server/klient, pak klient bude v Javě a server v C/C++.

\item Komunikace bude realizována textovým nešifrovaným protokolem nad TCP protokolem.

\item Výstupy serveru budou v alfanumerické podobě, klient může komunikovat i v grafice (není podmínkou).

\item Server řešte pod operačním systémem Linux, klient může běžet pod OS Windows XP. Emulátory typu Cygwin nebudou podporovány.

\item Realizujte konkurentní (paralelní) servery. Server musí být schopen obsluhovat požadavky více klientů souběžně.

\item Součástí programu bude trasování komunikace, dovolující zachytit proces komunikace na úrovni aplikačního protokolu a zápis trasování do souboru.

\item Každý program bude doplněn o zpracování statistických údajů (přenesený počet bytů, přenesený počet zpráv, počet navázaných spojení, počet přenosů zrušených pro chybu, doba běhu apod.).

\item Zdrojové kódy organizujte tak, aby od sebe byly odděleny části volání komunikačních funkcí, které jste vytvořili na základě zadání, od částí určených k demonstraci funkčnosti vašeho řešení (grafické rozhraní).
\end{itemize}

\section{Úvod}
Three Tunnelers je síťová adaptace rozšiřující možnosti originální hry Tunnelers pro DOS. V této hře hráč přebírá kontrolu nad hrabacím tankem v podzemní oblasti a jeho cílem je zabrat tuto oblast pro sebe. V oblasti se nachází další hráči, jejichž cílem je to samé. Zabrání oblasti je možné docílit tím, že bude hráč poslední žijící nejvíce-krát z daného počtu kol.

Hlavní vlastností hry Tunnelers je, že si hráči musejí postupně získat přehled o oblasti z pohledů na půdorys. K dispozici totiž nemají pohled na celou oblast ale jen na malé části kolem tanků. Každý hráč vidí půdorysy nad všemi tanky.


\section{Programátorská dokumentace}
\subsection{Datové struktury}
Pro zjednodušení popisu funkcí systémů jsou předpokládány následující struktury:
\subsubsection*{Mapa}
Mapa představuje oblast, o kterou se bojuje. Oblast se skládá z (W * H) sekcí, takzvaných chunků, stejné velikosti N. V oblasti je několik sekcí, které představují základnu.
\subsubsection*{Chunk}
Chunk je malý útvar skládající se z bloků, který reprezentuje terén oblasti. Některé chunky obsahují seskupení bloků představující hráčskou základnu, u těchto chunků je nastaven identifikátor odpovídajícího hráče.
\subsubsection*{Blok}
Blok na mapě je základní jednotku mapy. Jedná se pouze o číselnou hodnotu identifikující typ bloku, možné hodnoty, kterých blok může nabývat jsou: Prázdný blok = 0, Zemina = 1, Kámen = 2, Zeď základny = 3.
\subsubsection*{Klient}
Klient je struktura reprezentující klientskou aplikaci, pomocí které hráči vstupují a ovládají hru. Sdílené informace v této struktuře jsou klientské jméno a odkazy na hráče v místnosti.
\subsubsection*{Hráč}
Hráč je skupina informací o účastníkovi boje o oblast. Tato struktura je oddělena od Klienta pro jednodušší realizaci více hráčů v jedné spuštěné aplikaci. Struktura hráče obsahuje odkaz na klienta, kterému náleží, ovládání, což je struktura uchovávající stav stisknutých vstupů, a číslo barvy tohoto hráče.
\subsubsection*{Směr}
Směr je výčet identifikující jeden z 8 světových směrů reprezentovaný celočíselnou hodnotou. Možné hodnoty jsou: 0 = Žádný směr, 1 = Sever, 2 = Severovýchod, 3 = Východ, 4 = Jihovýchod, 5 = Jih, 6 = Jihozápad, 7 = Západ, 8 = Severozápad.
\subsubsection*{Tank}
Tank, rozlehlý 7*7 bloků, je struktura definující hráčem ovládaný prvek v oblasti. Tank je definován hráčem, kterému přísluší, svojí polohou na mapě a směrem, kam je natočen. Dále stav tanku definují jeho aktuální počet bodů zásahů, které zbývají do jeho zničení, zbývající energií a časem zbývajícím do možností vystřelení dalšího projektilu.
\subsubsection*{Projektil}
Projektil zabírající oblast pouze 3 bloky je stejně jako Tank definován hráčem, kterému náleží, svojí polohou na mapě a směrem kam je natočen a kam letí.
\subsubsection{Herní místnost}
Herní místnost je kontejner nesoucí informace aktuálním stavu místnosti, informace klientech, kteří se účastní dění v místnosti a identifikátor klienta, který má místnost "na starost". Dále místnost nese informace o hráčích účastnících se hry a v určitém stavu i o bitevní zóně.
Stavy místností jsou reprezentovány celočíselnými hodnotami 0 = Čeká, 1 = Lobby, 2 = Bitva začíná, 3 = Bitva, 4 = Souhrny.
\subsubsection*{Bitevní zóna}
Bitevní zóna je střed dění a zde se uchovávají informace o mapě definující oblast, o tancích v oblasti a o projektilech létajících oblastí.


\subsection{Herní server}
Zdrojové soubory serveru jsou tříděny do modulů podle jejich významu...
\subsection{Herní klient}
Zdrojové soubory serveru jsou tříděny do balíků podle jejich významu...
\section{Uživatelská dokumentace}

\section{Závěr}
Díky této semestrální práci jsem si vyzkoušel, co obnáší proces návrhu a implementace síťové aplikace skládající se ze serveru z klienta a aplikačního protokolu, pomocí kterého tyto aplikační části komunikují.
Procvičil jsem si vytváření uživatelského rozhraní a připomenul si důležitost očividné reakce na uživatelské vstupy.

Práci mi značně zjednodušily jednotkové testy, které jsem implementoval na straně klienta pomocí testovacího frameworku JUnit a na straně serveru, které jsou realizovány debugovacími výpisy. Obě aplikace nejsou ani zdaleka otestovány celé ale klíčové prvky lze testy ověřit.

V průběhu implementace protokolu na straně serveru v jazyce C jsem narazil na zvláštní chování při formátování proměnných typu \texttt{long int}, kde po předání této proměnné jako parametr do funkce a po vykonání dané funkce zmizela proměnná z paměti, čímž byla způsobena chyba typu SIGSEGF. Ještě pro jistotu se toto chování při několikanásobném spouštění testů občas neprojevila. Z toho důvodu jsem neimplementoval příkazy typu MARCO a POLO, které tento typ proměnné používají pro počítání latence.

Hra má veliký potenciál pro rozšířitelnost: například při implementaci přiřazování hráčských barev mě napadla možnost herních týmů, kde by určití hráči spolupracovali proti ostatním týmům.
\section{Přílohy}
\subsection{Aplikační protokol TTP}
Protokol TTP (Three Tunnelers Protocol) definuje rozhraní mezi dvěma účastníky komunikace po síti. Komunikace je asynchronní, což znamená že účastníci nemají stejné role, zde mají mezi sebou vztah Klient-Server.
Aplikační protokol STTP je realizován nad transportním protokolem TCP. Zprávy nemusí mít jednotnou délku ale vždy musí odpovídat následujícímu formátu:
\begin{center}
\texttt{<hlavička><tělo><LF>}
\end{center}
\begin{itemize}
\setlength\itemsep{0em}
\item \textbf{hlavička} zprávy se skládá ze 2 hexadecimálních znaků, které představují typ zprávy
\item \textbf{tělo} zprávy se odvíjí podle typu zprávy, čísla jsou přenášena v hexadecimální podobě a jsou vycpávána nulami na začátku do konstantní délky
\item \textbf{LF} představuje znak "Line Feed" a je vyžadován na konci každé zprávy
\end{itemize}

Protokol je bezestavový, což znamená, že žádná ze zpráv nevyžaduje okamžitou odpověď. Většina druhů zpráv však významově předpokládá návaznost jiných zpráv.

Definice zpráv protokolu podle hlaviček určují formát těla. V definici je možné se setkat s následujícími symboly:
\begin{itemize}
\item[C] = Znak
\item[X] = Znak představující hexadecimální hodnotu
\item[text]
\end{itemize}

\subsubsection*{Organizační zprávy mezi klientem a serverem}
\begin{description}
\item[1 Představení klienta] (XX)(text) [pořadí připojení, tajný kód] \\
Klient MUSÍ poslat jako první segment těla číselné pořadí určující, po kolikáté se klient přihlašuje. Rozpoznávané hodnoty jsou: nulová hodnota - klient se přihlašuje poprvé, nenulová hodnota - klient se snaží připojit znovu. Pokud se klient přihlašuje znovu, MUSÍ zahrnout druhý segment zprávy - tajný kód. \\
Server MUSÍ poslat jako první segment těla číselné pořadí, jako klient. V obou případech server MUSÍ zahrnout druhý segment - tajný kód.
Pokud se jedná o opětovné připojení klienta, server BY MĚL následovně poslat zprávu typu "Nastavení jména".
\item[2 Odpojení] (text)[důvod odpojení]\\
Pokud je to možné, klient i server BY MĚLI před ukončením spojení poslat zprávu o ukončení spojení. Zpráva MŮŽE obsahovat segment důvod odpojení.
\item[3 Marco] (XX)(XXXXXXXXXXXXXXXX)(XXXXXXXXXXXXXXXX) [počet předcházejících zpráv, časový otisk odesílající strany, časový otisk přijímající strany] \\
Zpráva vyžadující odpověď protistrany. Zpráva MUSÍ obsahovat časový otisk odesílající strany. Pokud je počet předcházejících zpráv nenulový, zpráva MUSÍ obsahovat i časový otisk přijímající strany. V opačném případě NESMÍ časový otisk přijímající strany. \\
Zpráva MŮŽE být použita jako odpověď, pokud přijímající strana také požaduje synchronizaci. V takovém případě MUSÍ být počet předcházejících zpráv nenulový.
\item[4 Polo] (XXXXXXXXXXXXXXXX) [Časový otisk přijímající strany]\\
Odpovídající zpráva na zprávu "Marco". Zpráva MUSÍ obsahovat časový otisk přijímající strany.
\item[5 Neplatný příkaz] (text) [přijatá neplatná zpráva]\\
Zpráva sloužící k informaci protistrany o neplatném formátu přijaté zprávy. Zpráva MUSÍ obsahovat neplatnou příchozí zprávu v podobě, v jaké byla přijata.
\item[10 Nastavení jména] (text) [Jméno klienta] \\
Zpráva od klienta MUSÍ obsahovat požadované jméno. Zpráva od serveru MUSÍ obsahovat jméno v takové podobě, v jaké je přijata.
\end{description}
\subsubsection*{Zprávy spravující místnosti}
\begin{description}
\item[14 Výpis místností] (XX)(M)*n [počet místností (n) ve zprávě, n popisů místnosti] \\
Zpráva od klienta NESMÍ obsahovat tělo. Zpráva od Serveru MUSÍ určit kolik místností nese a MUSÍ obsahovat tolik popisů místností. \\
Popis místnosti je tohoto formátu: (XXXX)(XX)(XX)(XX)(XX) [id místnosti, maximální počet hráčů, aktuální počet hráčů, stav místnosti, nastavená obtížnost]
\item[16 Vytvořit místnost] -
\item[17 Vstoupit do místnosti] (XX)(XX)(XX) [Číslo místnosti, klientské číslo v místnosti lokálního klienta, číslo klienta] \\ 
Klient MUSÍ poslat pouze číslo místnosti, do které se chce připojit.
\item[18 Opustit místnost] -
\end{description}
\subsubsection*{Mimo-herní zprávy v rámci místnost}
\begin{description}
\item[40 Zpráva - prostý text](XX)(text) [identifikátor odesílatele, tělo zprávy] \\
Klient NESMÍ zahrnovat identifikátor odesílatele.
\item[41 Systémová zpráva](text) [tělo zprávy] \\
Klient NESMÍ odesílat zprávy tohoto typu.
\item[42 Zpráva - vzdálený příkaz](text) [Příkaz k vykonání] \\
Umožňuje vzdálené vykonávání příkazů pro ovládání serveru jako z příkazové řádky.

\item[60 Synchronizace fáze] (XX) [stav místnosti]

\item[61 Stav připravení] (XX)(XX) [nová připravenost (nulová / nenulová hodnota), identifikátor klienta v místnosti]
Klient NESMÍ poslat identifikátor klienta.

\item[65 Informace o klientovi] (XX)(text)[identifikátor klienta v místnosti, jméno] \\
Klient MUSÍ poslat pouze identifikátor klienta. Server MUSÍ poskytnout k identifikátoru i kompletní popis klienta.

\item[66 Stav klienta] (XX)(XX)(XXXX) [identifikátor klienta v místnosti, stav klienta, doba odezvy v milisekundách] \\
Klient NESMÍ poslat tuto zprávu.

\item[67 Odebrání klienta] (XX)(text) [identifikátor klienta v místnosti, důvod odebrání klienta]

\item[68 Předat vedení klientovi] (XX) [číslo klienta v místnosti]
\end{description}

\subsubsection*{Herní zprávy místnosti}
\begin{description}
\item[80 Přidání hráče] (XX)(XX)(XX)(XX) [slot v místnosti, barva, číslo klienta v místnosti, číslo hráče klienta] \\
Klient NESMÍ zahrnout ve zprávě číslo klienta ani číslo hráče klienta. Klient NEMUSÍ zahrnout ve zprávě barvu.
\item[81 Odebrání hráče] (XX) [slot v místnosti]
\item[82 Přesunutí hráče mezi sloty] (XX)(XX) [slot odkud, slot kam]
\item[83 Nastavení barvy hráče] (XX)(XX) [slot hráče, barva]

\item[90 Specifikace mapy] (XX)(XX)(XX) [velikost chunku, šířka v chuncích, výška v chuncích]
\item[91 Specifikace základnových bloků na mapě] (XX)((XX)(XX)(XX)) * N [N - počet specifikací základen, N specifikací: (X chunku, Y chunku, číslo hráče v mínstnosti)]
\item[91 Chunková data] (XX)(XX)(XX)(text) [X chunku, Y chunku, checksum chunku, <domluvená velikost chunku> * X]\\
Klient NESMÍ posílat chunková data na server.
\item[92 Seznam změn bloků] (XX)(změna) * N [Počet změn bloků N, N změn bloků]\\ 
Změna bloku je následujícího formátu: (XXXX)(XXXX)(X) [X souřadnice bloku, Y souřadnice bloku, nový blok]

\item[120 Nastavení ovládání] (XX)(XX) - [číslo hráče v místnosti, bitová mapa stisklých kláves]
\item[130 Informace o tanku] (XX)(XXXX)(XXXX)(XX)(XX)(XX) [číslo hráče v místnosti, pozice X, pozice Y, směr, body zásahu, energie tanku] \\
Klient MUSÍ poslat pouze číslo hráče v místnosti.
\item[140 Vytvoření projektilu] (XX)(XX)(XX) [identifikátor projektilu, číslo hráče, počáteční směr]
\item[141 Odebrání projektilu] (XX) [identifikátor projektilu]
\end{description}




%------------------------------------------

\end{document}
